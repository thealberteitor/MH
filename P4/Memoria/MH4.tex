\input{preambuloSimple.tex}

\usepackage{algpseudocode}
\usepackage[spanish]{babel}
\usepackage{varwidth}
\usepackage{hyperref}
\usepackage{graphicx}
\usepackage{epstopdf}


\selectlanguage{spanish} 
\usepackage[spanish,onelanguage]{algorithm2e} %for psuedo code
\usepackage[lmargin=3.81cm,tmargin=2.54cm,rmargin=2.54cm,bmargin=2.52cm]{geometry}
%\usepackage{listings}
%\usepackage{dsfont}
\usepackage{movie15}
%----------------------------------------------------------------------------------------
%	TÍTULO Y DATOS DEL ALUMNO
%----------------------------------------------------------------------------------------

\title{	
\normalfont \normalsize 
\textsc{\textbf{Metaheurísticas(2019-2020)} \\ Doble Grado en Ingeniería Informática y Matemáticas \\ Universidad de Granada} \\ [25pt] % Your university, school and/or department name(s)
\horrule{0.5pt} \\[0.4cm] % Thin top horizontal rule
\huge Propuesta de metaheurística personalizada \\ para el Problema de la Máxima Diversidad  (MDP) \\ % The assignment title
\horrule{2pt} \\[0.5cm] % Thick bottom horizontal rule
}
\author{Alberto Jesús Durán López \\ 
DNI: 54142189-M \\
albduranlopez@gmail.com \\
\hfill \break \hspace{1cm}\\
Grupo Jueves, 17:30 - 19:30 } % Nombre y apellidos


\date{\normalsize\today} % Incluye la fecha actual

%----------------------------------------------------------------------------------------
% DOCUMENTO
%----------------------------------------------------------------------------------------

\begin{document}

\maketitle % Muestra el Título

\newpage %inserta un salto de página

\tableofcontents % para generar el índice de contenidos

% \listoffigures

% \listoftables

\newpage


\section{Introducción}
\hspace{1.5cm} Durante el transcurso de la asignatura se ha trabajado con el problema de la máxima diversidad \textit{(Max Diversity Problem)}. 


En particular, en esta práctica estudiaremos técnicas de búsqueda basadas en poblaciones para la resolución del problema en cuestión. \\
Comentaremos todos los pasos y problemas encontrados, detallando minuciosamente todos los detalles y soluciones a los mismos. \\

Además, se incorporarán tablas para mostrar los resultados de todas las ejecuciones y  gráficas para contrastar los modelos (optimización, costes y desviación).


\section{Problema de la máxima Diversidad}
\hspace{1.5cm} El problema de la máxima diversidad (\textit{maximum diversity problem}, MDP) es un problema de optimización combinatoria consistente en seleccionar un subconjunto de m elementos ($|M|=m$) de un conjunto inicial N de n elementos (con $n>m$) de forma que se maximice la diversidad entre los elementos escogidos. \\


El \textbf{MDP} se puede formular como: \\


\[
\text{Maximizar:   }  z_{MS}(x) = \sum_{i=1}^{n-1} \sum_{j=i+1}^{n} d_{ij}x_i x_j
\]



\[
\text{Sujeto a: } \sum_{i=1}^{n} x_i=m \hspace{0.2cm} \text{ con } x_i=\{0,1\} \text{, } i=1,...,n  \hspace{0.5cm} \text{donde: }
\]


\begin{itemize}
	\item x es una solución al problema que consiste en un vector binario que indica los m elementos seleccionados.
	\item $d_{ij}$ es la distancia existente entre los elementos $i$ y $j$
	
\end{itemize}


\begin{figure}[H]
	\centering
	\includegraphics[scale=0.35]{img/mdp.png}
	\caption{MDP-Maximum Diversity Problem}
\end{figure}


\newpage
\section{Datos y casos considerados}
\hspace{1.5cm} Las ejecuciones se han realizado en un ordenador \texttt{Intel(R) Core(TM) i7-1065G7 CPU @ 1.30GHz, 16GB RAM, 512 SSD}. 


Se utilizarán \textbf{30 casos} seleccionados de varios de los conjuntos de instancias disponible en la \href{http://www.optsicom.es/mdp/}{MDPLIB}, 10 pertenecientes al grupo \texttt{GKD} con distancias Euclideas, n=500 y m=50 \textit{(GKD-c\_11\_n500\_m50 a GKD-
c\_20\_n500\_m50)}, 10 del grupo \texttt{MDG} con distancias reales en [0,1000], n=500 y m=50
\textit{(MDG-b\_1\_n500\_m50 a MDG-b\_10\_n500\_m50)}; y otras 10 del grupo \texttt{MDG} con distancias enteras en {0,10}, n=2000 y m=200 \textit{(MDG-a\_31\_n2000\_m200 a MDG-
a\_40\_n2000\_m200)}. \\


Para la realización de la práctica usaremos el lenguaje de programación \texttt{C++} ya que se deben probar muchos ejemplos y la ejecución es más rápida al ser un lenguaje de programación compilado. \\
Todos los archivos se han compilado con la opción de optimización \texttt{-O2} y, además,  la semilla usada para todas
las ejecuciones ha sido 54142189, pudiéndose ésta indicar en la ejecución del archivo en cuestión. 


\section{Metaheurística original propuesta}

Los juegos clásicos que todos conocemos como el buscaminas o el pong, entre otros, se han llevado a cabo en un tablero finito  de dimensiones NXM. 
Sin embargo, existen otros juegos como el snake cuyo tablero no es un plano finito, es decir, se juega sobre un toro llano. SUper ficie no única ,se puede dotar a estos tableros de identificaciones para modificar el tablero.

Antes de empezar, distinguimos los dos tipos de superficies que existen:

\begin{itemize}
	\item  Una superficie es no orientable si existe al menos una curva cerrada simple contenida que es homeomorfa a una banda de Möbius. Ejemplos: La propia banda de Möbius, plano proyectivo, botella de Klein..etc
	\item En caso contrario, la superficie será orientable. Ejemplos: Toro llano, suma conexa de k toros,
\end{itemize}




\newpage 
\section{Esquema de representación de soluciones}

Para todos los algoritmos menos para la \textit{Búsqueda Local}, se ha usado una codificación basada en números binarios. Sin embargo, dejaremos la explicación para las siguientes secciones, donde se explicará con detalle en cada algoritmo.


\section{Descripción de la Función Objetivo}

Todos los algoritmos implementados hacen uso de la misma función objetivo, la comentada en el punto (2) anterior y cuyo pseudocódigo mostramos a continuación. Bien es cierto que en el caso de la \textit{Búsqueda Local} se tiene que usar codificación basada en números enteros, sin embargo, dicha función se ve mínimamente modificada.


\begin{figure}[H]
	\centering
	\begin{minipage}{.9\linewidth}
		
		
		
		\begin{algorithm}[H] 
			

			\SetAlgoLined
			
			\KwData{\textbf{ContribucionIndep}(\textit{integer ind, vector sel, matriz distancias)}}
			
			\Begin{
				\textit{suma $\leftarrow$ 0} \;
				
				\For{j in sel}{
					$ suma \leftarrow sel[j]\cdot(distancias[ind][j] + suma)$ \;	
				}
				
				\Return $suma$
			}
			
		\end{algorithm} 
		
	\end{minipage}
\end{figure}



\begin{figure}[H]
	\centering
	\begin{minipage}{.9\linewidth}
		
		
		
		\begin{algorithm}[H] 
			
			\caption{Evaluar solución}
			\SetAlgoLined
			
			\KwData{\textbf{CosteEstimado}(\textit{vector sel, matriz distancias)}}
			
			\Begin{
				
				\textit{suma $\leftarrow$ 0} \;
				
				
				\For{i in $|sel|$}{
					\If{$sel[i]=true$}{
						$ suma \leftarrow ContribucionIndep(i, sel, distancias) + suma$ \;
					}
					
				}
				
				\Return $suma/2$
			}
			
		\end{algorithm} 
		
	\end{minipage}
\end{figure}




\newpage 

\section{Explotación Multi-Entorno}



\begin{figure}[H]
	\centering
	\begin{minipage}{.75\linewidth}
		
		\begin{algorithm}[H] 
			\caption{Mundo Superficie }
			\SetAlgoLined
			
			\KwData{\textbf{MundoSuperficie}}
			
			
			\Begin{
				$Superficies \leftarrow $ \{\textit{Botella de Klein, Plano proyectivo,} \hspace*{2.4cm}  \textit{Toro llano} \} \;
				
				\hfill \break
				\For{$s \in Superficies$}{
					$s \leftarrow$ \textbf{SolucionInicial(n,m)} \;
					
				}
				\While{$eval < n\_eval$}{
					\For{$s \in Superficies$}{
						$movimiento$ \;
						%$evaluarPoblacion$ \;
						$nutrirse$ \;
						$mudar\_piel$ \;	
					}
				
					\If{\textit{se dan las condiciones para viajar}}{
						\For{$s, s' \in Superficies$}{
							%\textit{aplicar\_BL a los} $\frac{n}{2}$ \textit{peores de la población de s} \;
							\textit{s viaja a s'}\;
						}	
					}
				}
			}
		\end{algorithm}
	\end{minipage}
\end{figure}











\newpage 
\section{Hibridación con Búsqueda Local}
\hspace{1.5cm}El algoritmo de \textit{Búsqueda Local} usado para nuestro algoritmo se corresponde con el implementado en la primera práctica, por ello, conviene recordar como funciona: 

\begin{itemize}
	
	
	
	\item Se ha implementado una función de intercambio, Int(sel,i,j). El parámetro \texttt{i} indica el índice del elemento que se eliminará de la solución y el j por cual se sustituirá. Por tanto, las opciones para \texttt{i} serán \texttt{m} (los m elementos seleccionados) y las opciones para \texttt{j} serán \texttt{n-m} (los elementos disponibles en \textit{no seleccionados}), por tanto, el entorno estará formado por \texttt{m$\cdot$(n-m)} elementos.
	
	
	
	\begin{figure}[H]
		\centering
		\begin{minipage}{.75\linewidth}
			
			\begin{algorithm}[H] 
				\caption{EvaluaVecinos }
				\SetAlgoLined
				
				\KwData{\textbf{EvaluaVecinos}}
				
				
				\Begin{
					
					$mejora \leftarrow True$\;
					$eval \leftarrow 0$ \;
					
					\break \hfill
					
					
					\While{\textit{eval<100.000 and mejora}}{
						
						
						
						\break \hfill
						
						\hspace{3cm}\textbf{. . .}
						
						
						
						\break \hfill
						
						$salir \leftarrow$ False, $c \leftarrow 0$ \;
						
						\While{$c< n-m$ and not salir}{
							
							\textit{Genera elemento válido j} \;
							
							\If{\texttt{Contrib($s_j$)}>\texttt{Contrib($s_i$)}}{
								
								\textit{Int(sel, i, j)} \;
								$salir \leftarrow$ True\;
								$coste\leftarrow coste + contribNueva - contribAntigua$ 
							}
							$c\leftarrow c+1$\;
							$eval\leftarrow eval+1$\;
						}
					}
				}
			\end{algorithm}
		\end{minipage}
	\end{figure}
	
	
	\item En cada iteración, podemos calcular la diferencia de costes entre las dos soluciones recalculando todas las distancias de la función objetivo, sin embargo, esto no es necesario ya que como únicamente añadimos y quitamos 1 elemento, basta con sumar y restar la distancia del nuevo y del viejo elemento al resto de elementos seleccionados, respectivamente. Además, combinamos la factorización del coste con el cálculo de la contribución de los elementos para mejorar aún más la eficiencia.
	
	
	
	Si el intercambio es favorable, es decir, si el coste de la nueva solución (sumando las distancias del nuevo elemento y restando las del elemento anterior) es mayor que la anterior, aceptamos el intercambio. En caso contrario, rechazamos.
	
	\item Repetiremos este proceso hasta que se realicen \textit{N} evaluaciones de la función objetivo o cuando no encuentre mejora en el entorno.
	
\end{itemize}







\newpage
\section{Búsqueda Tabú}

La Búsqueda Tabú es una metaheurística que utiliza un procedimiento de búsqueda heurística local para explorar el espacio de la solución que pertenece a un óptimo local. Uno de los principales características de este método es el uso de memoria adaptativa, que hace que el comportamiento de búsqueda sea más flexible.

Usaremos dos listas:
\begin{itemize}
	\item \texttt{freq}: se almacena el número de veces que el elemento $s_i$ ha sido seleccionado en construcciones previas donde $max\_freq$ representa el máximo valor de $freq[i]$ $\forall i$, $i \in {0,...,n}$
	
	\item \texttt{quality}: En esta lista se almacena la media de las soluciones previas en las que el elemento $s_i$ ha sido seleccionado. Por otro lado, $max\_q$ hace referencia al máximo valor de $quality[i]$ $\forall i$, $i \in {0,...,n}$
\end{itemize}

Then, we modify the evaluation of the attractiveness of each non-
selected element in the current construction according to these quantities to favor the selection of
elements with low frequency and high quality values.


Mostramos a continuación el pseudocódigo asociado a la función principal de nuestro algoritmo donde S representa el número total de índices, Sel los índices seleccionados y S-Sel los no seleccionados.

\begin{figure}[H]
	\centering
	\begin{minipage}{.95\linewidth}
		
		
		
		\begin{algorithm}[H] 
			
			\caption{Búsqueda Tabú}
			\SetAlgoLined
			
			\KwData{\textbf{BusquedaTabu}}
			
			\Begin{
				
				
				$freq[i] = quality[i] $ $\forall s_i \in S$ \;
				
				
				\For{$eval < n\_eval$}{
					$sel \leftarrow \emptyset$ \;
					$s_c = s\_center(S)$ \;
					\While{$|sel|<m$}{
						\For{$s_i \in S-sel$}{
							$d'(s_i,s_c) \leftarrow d(s_i,s_c)-\beta d(s_i,s_c)\frac{freq[i]}{max\_freq}+\delta d(s_i,s_c)\frac{quality[i]}{max\_q}$ \;
						}
						\textit{Tomar $i^*$ tal que} $d'(s_{i^*},s_c)=max\{d'(s_i,s_c)\}$ \;
						$freq[i^*] \leftarrow freq[i^*]+1 $\;
						$sel \leftarrow sel \cup s_{i^*}$ \;
						$S \leftarrow S-s_{i^*}$\;
						$s_c \leftarrow s\_center(sel)$\;
					}
					$z \leftarrow Coste$$Solucion(sel)$ \;
					$quality[i] \leftarrow \frac{quality[i](freq[i]-1)+z}{freq[i]}$ $\forall s_i
					\in sel$ \;
				}
				
			}
			
		\end{algorithm} 
		
	\end{minipage}
\end{figure}

\newpage











\section{Resultados Obtenidos}













% Please add the following required packages to your document preamble:
% \usepackage{booktabs}
% \usepackage{graphicx}
% \usepackage[table,xcdraw]{xcolor}
% If you use beamer only pass "xcolor=table" option, i.e. \documentclass[xcolor=table]{beamer}
\begin{table}[H]
	\centering
	\resizebox{\textwidth}{!}{%
		\begin{tabular}{@{}
				>{\columncolor[HTML]{FFFFFF}}c 
				>{\columncolor[HTML]{FFFFFF}}c 
				>{\columncolor[HTML]{FFFFFF}}c 
				>{\columncolor[HTML]{FFFFFF}}c 
				>{\columncolor[HTML]{FFFFFF}}c 
				>{\columncolor[HTML]{FFFFFF}}c 
				>{\columncolor[HTML]{FFFFFF}}c 
				>{\columncolor[HTML]{FFFFFF}}c @{}}
			\toprule
			{\color[HTML]{333333} \textbf{Instancia}}     & {\color[HTML]{333333} \textbf{Mejor Coste}} & {\color[HTML]{333333} \textbf{ES}} & {\color[HTML]{333333} \textbf{Desv}} & {\color[HTML]{333333} \textbf{Time}} & {\color[HTML]{333333} \textbf{BMB}} & {\color[HTML]{333333} \textbf{Desv}} & {\color[HTML]{333333} \textbf{Time}} \\ \midrule
			{\color[HTML]{333333} GKD-c\_11\_n500\_m50}   & {\color[HTML]{333333} 19587,13}             & {\color[HTML]{333333} 19581,2}     & {\color[HTML]{333333} 0,03}          & {\color[HTML]{333333} 0,03}          & {\color[HTML]{333333} 19586}        & {\color[HTML]{333333} 0,01}          & {\color[HTML]{333333} 0,03}          \\
			{\color[HTML]{333333} GKD-c\_12\_n500\_m50}   & {\color[HTML]{333333} 19360,24}             & {\color[HTML]{333333} 19357}       & {\color[HTML]{333333} 0,02}          & {\color[HTML]{333333} 0,04}          & {\color[HTML]{333333} 19360,2}      & {\color[HTML]{333333} 0,00}          & {\color[HTML]{333333} 0,05}          \\
			{\color[HTML]{333333} GKD-c\_13\_n500\_m50}   & {\color[HTML]{333333} 19366,70}             & {\color[HTML]{333333} 19366,7}     & {\color[HTML]{333333} 0,00}          & {\color[HTML]{333333} 0,03}          & {\color[HTML]{333333} 19366,7}      & {\color[HTML]{333333} 0,00}          & {\color[HTML]{333333} 0,04}          \\
			{\color[HTML]{333333} GKD-c\_14\_n500\_m50}   & {\color[HTML]{333333} 19458,56}             & {\color[HTML]{333333} 19453,1}     & {\color[HTML]{333333} 0,03}          & {\color[HTML]{333333} 0,04}          & {\color[HTML]{333333} 19458,6}      & {\color[HTML]{333333} 0,00}          & {\color[HTML]{333333} 0,04}          \\
			{\color[HTML]{333333} GKD-c\_15\_n500\_m50}   & {\color[HTML]{333333} 19422,15}             & {\color[HTML]{333333} 19418,6}     & {\color[HTML]{333333} 0,02}          & {\color[HTML]{333333} 0,03}          & {\color[HTML]{333333} 19421,6}      & {\color[HTML]{333333} 0,00}          & {\color[HTML]{333333} 0,03}          \\
			{\color[HTML]{333333} GKD-c\_16\_n500\_m50}   & {\color[HTML]{333333} 19680,21}             & {\color[HTML]{333333} 19665,2}     & {\color[HTML]{333333} 0,08}          & {\color[HTML]{333333} 0,02}          & {\color[HTML]{333333} 19680,2}      & {\color[HTML]{333333} 0,00}          & {\color[HTML]{333333} 0,04}          \\
			{\color[HTML]{333333} GKD-c\_17\_n500\_m50}   & {\color[HTML]{333333} 19331,39}             & {\color[HTML]{333333} 19327,3}     & {\color[HTML]{333333} 0,02}          & {\color[HTML]{333333} 0,03}          & {\color[HTML]{333333} 19331,4}      & {\color[HTML]{333333} 0,00}          & {\color[HTML]{333333} 0,03}          \\
			{\color[HTML]{333333} GKD-c\_18\_n500\_m50}   & {\color[HTML]{333333} 19461,39}             & {\color[HTML]{333333} 19459,8}     & {\color[HTML]{333333} 0,01}          & {\color[HTML]{333333} 0,03}          & {\color[HTML]{333333} 19461,4}      & {\color[HTML]{333333} 0,00}          & {\color[HTML]{333333} 0,03}          \\
			{\color[HTML]{333333} GKD-c\_19\_n500\_m50}   & {\color[HTML]{333333} 19477,39}             & {\color[HTML]{333333} 19471,6}     & {\color[HTML]{333333} 0,03}          & {\color[HTML]{333333} 0,04}          & {\color[HTML]{333333} 19477,3}      & {\color[HTML]{333333} 0,00}          & {\color[HTML]{333333} 0,04}          \\
			{\color[HTML]{333333} GKD-c\_20\_n500\_m50}   & {\color[HTML]{333333} 19604,84}             & {\color[HTML]{333333} 19601,8}     & {\color[HTML]{333333} 0,02}          & {\color[HTML]{333333} 0,05}          & {\color[HTML]{333333} 19604,8}      & {\color[HTML]{333333} 0,00}          & {\color[HTML]{333333} 0,04}          \\
			{\color[HTML]{333333} MDG-b\_1\_n500\_m50}    & {\color[HTML]{333333} 778030,62}            & {\color[HTML]{333333} 762432}      & {\color[HTML]{333333} 2,00}          & {\color[HTML]{333333} 0,07}          & {\color[HTML]{333333} 768472}       & {\color[HTML]{333333} 1,23}          & {\color[HTML]{333333} 0,04}          \\
			{\color[HTML]{333333} MDG-b\_2\_n500\_m50}    & {\color[HTML]{333333} 779963,69}            & {\color[HTML]{333333} 764288}      & {\color[HTML]{333333} 2,01}          & {\color[HTML]{333333} 0,05}          & {\color[HTML]{333333} 771856}       & {\color[HTML]{333333} 1,04}          & {\color[HTML]{333333} 0,04}          \\
			{\color[HTML]{333333} MDG-b\_3\_n500\_m50}    & {\color[HTML]{333333} 776768,44}            & {\color[HTML]{333333} 763946}      & {\color[HTML]{333333} 1,65}          & {\color[HTML]{333333} 0,05}          & {\color[HTML]{333333} 767181}       & {\color[HTML]{333333} 1,23}          & {\color[HTML]{333333} 0,04}          \\
			{\color[HTML]{333333} MDG-b\_4\_n500\_m50}    & {\color[HTML]{333333} 775394,62}            & {\color[HTML]{333333} 766680}      & {\color[HTML]{333333} 1,12}          & {\color[HTML]{333333} 0,06}          & {\color[HTML]{333333} 765863}       & {\color[HTML]{333333} 1,23}          & {\color[HTML]{333333} 0,04}          \\
			{\color[HTML]{333333} MDG-b\_5\_n500\_m50}    & {\color[HTML]{333333} 775611,06}            & {\color[HTML]{333333} 753849}      & {\color[HTML]{333333} 2,81}          & {\color[HTML]{333333} 0,02}          & {\color[HTML]{333333} 770415}       & {\color[HTML]{333333} 0,67}          & {\color[HTML]{333333} 0,04}          \\
			{\color[HTML]{333333} MDG-b\_6\_n500\_m50}    & {\color[HTML]{333333} 775153,69}            & {\color[HTML]{333333} 759891}      & {\color[HTML]{333333} 1,97}          & {\color[HTML]{333333} 0,05}          & {\color[HTML]{333333} 765810}       & {\color[HTML]{333333} 1,21}          & {\color[HTML]{333333} 0,04}          \\
			{\color[HTML]{333333} MDG-b\_7\_n500\_m50}    & {\color[HTML]{333333} 777232,87}            & {\color[HTML]{333333} 762001}      & {\color[HTML]{333333} 1,96}          & {\color[HTML]{333333} 0,04}          & {\color[HTML]{333333} 766665}       & {\color[HTML]{333333} 1,36}          & {\color[HTML]{333333} 0,04}          \\
			{\color[HTML]{333333} MDG-b\_8\_n500\_m50}    & {\color[HTML]{333333} 779168,75}            & {\color[HTML]{333333} 764350}      & {\color[HTML]{333333} 1,90}          & {\color[HTML]{333333} 0,05}          & {\color[HTML]{333333} 773351}       & {\color[HTML]{333333} 0,75}          & {\color[HTML]{333333} 0,04}          \\
			{\color[HTML]{333333} MDG-b\_9\_n500\_m50}    & {\color[HTML]{333333} 774802,19}            & {\color[HTML]{333333} 756551}      & {\color[HTML]{333333} 2,36}          & {\color[HTML]{333333} 0,03}          & {\color[HTML]{333333} 766528}       & {\color[HTML]{333333} 1,07}          & {\color[HTML]{333333} 0,03}          \\
			{\color[HTML]{333333} MDG-b\_10\_n500\_m50}   & {\color[HTML]{333333} 774961,31}            & {\color[HTML]{333333} 759213}      & {\color[HTML]{333333} 2,03}          & {\color[HTML]{333333} 0,05}          & {\color[HTML]{333333} 771090}       & {\color[HTML]{333333} 0,50}          & {\color[HTML]{333333} 0,04}          \\
			{\color[HTML]{333333} MDG-a\_31\_n2000\_m200} & {\color[HTML]{333333} 114139}               & {\color[HTML]{333333} 112544}      & {\color[HTML]{333333} 1,40}          & {\color[HTML]{333333} 0,41}          & {\color[HTML]{333333} 113028}       & {\color[HTML]{333333} 0,97}          & {\color[HTML]{333333} 0,67}          \\
			{\color[HTML]{333333} MDG-a\_32\_n2000\_m200} & {\color[HTML]{333333} 114092}               & {\color[HTML]{333333} 112382}      & {\color[HTML]{333333} 1,50}          & {\color[HTML]{333333} 0,50}          & {\color[HTML]{333333} 112651}       & {\color[HTML]{333333} 1,26}          & {\color[HTML]{333333} 0,65}          \\
			{\color[HTML]{333333} MDG-a\_33\_n2000\_m200} & {\color[HTML]{333333} 114124}               & {\color[HTML]{333333} 112426}      & {\color[HTML]{333333} 1,49}          & {\color[HTML]{333333} 0,49}          & {\color[HTML]{333333} 112944}       & {\color[HTML]{333333} 1,03}          & {\color[HTML]{333333} 0,65}          \\
			{\color[HTML]{333333} MDG-a\_34\_n2000\_m200} & {\color[HTML]{333333} 114203}               & {\color[HTML]{333333} 112573}      & {\color[HTML]{333333} 1,43}          & {\color[HTML]{333333} 0,43}          & {\color[HTML]{333333} 112835}       & {\color[HTML]{333333} 1,20}          & {\color[HTML]{333333} 0,63}          \\
			{\color[HTML]{333333} MDG-a\_35\_n2000\_m200} & {\color[HTML]{333333} 114180}               & {\color[HTML]{333333} 112820}      & {\color[HTML]{333333} 1,19}          & {\color[HTML]{333333} 0,44}          & {\color[HTML]{333333} 112944}       & {\color[HTML]{333333} 1,08}          & {\color[HTML]{333333} 0,64}          \\
			{\color[HTML]{333333} MDG-a\_36\_n2000\_m200} & {\color[HTML]{333333} 114252}               & {\color[HTML]{333333} 112386}      & {\color[HTML]{333333} 1,63}          & {\color[HTML]{333333} 0,47}          & {\color[HTML]{333333} 112987}       & {\color[HTML]{333333} 1,11}          & {\color[HTML]{333333} 0,64}          \\
			{\color[HTML]{333333} MDG-a\_37\_n2000\_m200} & {\color[HTML]{333333} 114213}               & {\color[HTML]{333333} 112244}      & {\color[HTML]{333333} 1,72}          & {\color[HTML]{333333} 0,43}          & {\color[HTML]{333333} 112874}       & {\color[HTML]{333333} 1,17}          & {\color[HTML]{333333} 0,65}          \\
			{\color[HTML]{333333} MDG-a\_38\_n2000\_m200} & {\color[HTML]{333333} 114378}               & {\color[HTML]{333333} 112921}      & {\color[HTML]{333333} 1,27}          & {\color[HTML]{333333} 0,40}          & {\color[HTML]{333333} 112991}       & {\color[HTML]{333333} 1,21}          & {\color[HTML]{333333} 0,65}          \\
			{\color[HTML]{333333} MDG-a\_39\_n2000\_m200} & {\color[HTML]{333333} 114201}               & {\color[HTML]{333333} 112209}      & {\color[HTML]{333333} 1,74}          & {\color[HTML]{333333} 0,41}          & {\color[HTML]{333333} 112908}       & {\color[HTML]{333333} 1,13}          & {\color[HTML]{333333} 0,64}          \\
			{\color[HTML]{333333} MDG-a\_40\_n2000\_m200} & {\color[HTML]{333333} 114191}               & {\color[HTML]{333333} 112819}      & {\color[HTML]{333333} 1,20}          & {\color[HTML]{333333} 0,40}          & {\color[HTML]{333333} 112834}       & {\color[HTML]{333333} 1,19}          & {\color[HTML]{333333} 0,65}          \\ \bottomrule
		\end{tabular}%
	}
	\caption{Resultados ES/BMB}
	\label{AMallo2}
\end{table}








% Please add the following required packages to your document preamble:
% \usepackage{booktabs}
% \usepackage{graphicx}
% \usepackage[table,xcdraw]{xcolor}
% If you use beamer only pass "xcolor=table" option, i.e. \documentclass[xcolor=table]{beamer}
\begin{table}[H]
	\centering
	\resizebox{\textwidth}{!}{%
		\begin{tabular}{@{}
				>{\columncolor[HTML]{FFFFFF}}c 
				>{\columncolor[HTML]{FFFFFF}}c 
				>{\columncolor[HTML]{FFFFFF}}c 
				>{\columncolor[HTML]{FFFFFF}}c 
				>{\columncolor[HTML]{FFFFFF}}c 
				>{\columncolor[HTML]{FFFFFF}}c 
				>{\columncolor[HTML]{FFFFFF}}c 
				>{\columncolor[HTML]{FFFFFF}}c @{}}
			\toprule
			{\color[HTML]{333333} \textbf{Instancia}}     & {\color[HTML]{333333} \textbf{Mejor Coste}} & {\color[HTML]{330001} \textbf{ILS}} & {\color[HTML]{330001} \textbf{Desv}} & {\color[HTML]{330001} \textbf{Time}} & {\color[HTML]{330001} \textbf{ILS-ES}} & {\color[HTML]{330001} \textbf{Desv}} & {\color[HTML]{330001} \textbf{Time}} \\ \midrule
			{\color[HTML]{330001} GKD-c\_11\_n500\_m50}   & {\color[HTML]{330001} 19587,13}             & {\color[HTML]{333333} 19587,1}      & {\color[HTML]{333333} 0,00}          & {\color[HTML]{333333} 0,03}          & {\color[HTML]{333333} 19584,6}         & {\color[HTML]{333333} 0,01}          & {\color[HTML]{333333} 0,31}          \\
			{\color[HTML]{330001} GKD-c\_12\_n500\_m50}   & {\color[HTML]{330001} 19360,24}             & {\color[HTML]{333333} 19360,2}      & {\color[HTML]{333333} 0,00}          & {\color[HTML]{333333} 0,04}          & {\color[HTML]{333333} 19360,2}         & {\color[HTML]{333333} 0,00}          & {\color[HTML]{333333} 0,32}          \\
			{\color[HTML]{330001} GKD-c\_13\_n500\_m50}   & {\color[HTML]{330001} 19366,70}             & {\color[HTML]{333333} 19366,7}      & {\color[HTML]{333333} 0,00}          & {\color[HTML]{333333} 0,05}          & {\color[HTML]{333333} 19366,7}         & {\color[HTML]{333333} 0,00}          & {\color[HTML]{333333} 0,32}          \\
			{\color[HTML]{330001} GKD-c\_14\_n500\_m50}   & {\color[HTML]{330001} 19458,56}             & {\color[HTML]{333333} 19458,6}      & {\color[HTML]{333333} 0,00}          & {\color[HTML]{333333} 0,03}          & {\color[HTML]{333333} 19458,6}         & {\color[HTML]{333333} 0,00}          & {\color[HTML]{333333} 0,29}          \\
			{\color[HTML]{330001} GKD-c\_15\_n500\_m50}   & {\color[HTML]{330001} 19422,15}             & {\color[HTML]{333333} 19421,6}      & {\color[HTML]{333333} 0,00}          & {\color[HTML]{333333} 0,03}          & {\color[HTML]{333333} 19422,1}         & {\color[HTML]{333333} 0,00}          & {\color[HTML]{333333} 0,33}          \\
			{\color[HTML]{330001} GKD-c\_16\_n500\_m50}   & {\color[HTML]{330001} 19680,21}             & {\color[HTML]{333333} 19680,2}      & {\color[HTML]{333333} 0,00}          & {\color[HTML]{333333} 0,03}          & {\color[HTML]{333333} 19680,2}         & {\color[HTML]{333333} 0,00}          & {\color[HTML]{333333} 0,32}          \\
			{\color[HTML]{330001} GKD-c\_17\_n500\_m50}   & {\color[HTML]{330001} 19331,39}             & {\color[HTML]{333333} 19331,4}      & {\color[HTML]{333333} 0,00}          & {\color[HTML]{333333} 0,04}          & {\color[HTML]{333333} 19329,4}         & {\color[HTML]{333333} 0,01}          & {\color[HTML]{333333} 0,29}          \\
			{\color[HTML]{330001} GKD-c\_18\_n500\_m50}   & {\color[HTML]{330001} 19461,39}             & {\color[HTML]{333333} 19461,4}      & {\color[HTML]{333333} 0,00}          & {\color[HTML]{333333} 0,04}          & {\color[HTML]{333333} 19461,4}         & {\color[HTML]{333333} 0,00}          & {\color[HTML]{333333} 0,35}          \\
			{\color[HTML]{330001} GKD-c\_19\_n500\_m50}   & {\color[HTML]{330001} 19477,39}             & {\color[HTML]{333333} 19477,3}      & {\color[HTML]{333333} 0,00}          & {\color[HTML]{333333} 0,03}          & {\color[HTML]{333333} 19477,3}         & {\color[HTML]{333333} 0,00}          & {\color[HTML]{333333} 0,32}          \\
			{\color[HTML]{330001} GKD-c\_20\_n500\_m50}   & {\color[HTML]{330001} 19604,84}             & {\color[HTML]{333333} 19604,8}      & {\color[HTML]{333333} 0,00}          & {\color[HTML]{333333} 0,04}          & {\color[HTML]{333333} 19604,8}         & {\color[HTML]{333333} 0,00}          & {\color[HTML]{333333} 0,34}          \\
			{\color[HTML]{330001} MDG-b\_1\_n500\_m50}    & {\color[HTML]{330001} 778030,62}            & {\color[HTML]{333333} 768893}       & {\color[HTML]{333333} 1,17}          & {\color[HTML]{333333} 0,04}          & {\color[HTML]{333333} 770455}          & {\color[HTML]{333333} 0,97}          & {\color[HTML]{333333} 0,59}          \\
			{\color[HTML]{330001} MDG-b\_2\_n500\_m50}    & {\color[HTML]{330001} 779963,69}            & {\color[HTML]{333333} 776145}       & {\color[HTML]{333333} 0,49}          & {\color[HTML]{333333} 0,04}          & {\color[HTML]{333333} 768793}          & {\color[HTML]{333333} 1,43}          & {\color[HTML]{333333} 0,38}          \\
			{\color[HTML]{330001} MDG-b\_3\_n500\_m50}    & {\color[HTML]{330001} 776768,44}            & {\color[HTML]{333333} 769205}       & {\color[HTML]{333333} 0,97}          & {\color[HTML]{333333} 0,05}          & {\color[HTML]{333333} 768283}          & {\color[HTML]{333333} 1,09}          & {\color[HTML]{333333} 0,65}          \\
			{\color[HTML]{330001} MDG-b\_4\_n500\_m50}    & {\color[HTML]{330001} 775394,62}            & {\color[HTML]{333333} 769641}       & {\color[HTML]{333333} 0,74}          & {\color[HTML]{333333} 0,04}          & {\color[HTML]{333333} 767491}          & {\color[HTML]{333333} 1,02}          & {\color[HTML]{333333} 0,45}          \\
			{\color[HTML]{330001} MDG-b\_5\_n500\_m50}    & {\color[HTML]{330001} 775611,06}            & {\color[HTML]{333333} 769000}       & {\color[HTML]{333333} 0,85}          & {\color[HTML]{333333} 0,03}          & {\color[HTML]{333333} 765396}          & {\color[HTML]{333333} 1,32}          & {\color[HTML]{333333} 0,38}          \\
			{\color[HTML]{330001} MDG-b\_6\_n500\_m50}    & {\color[HTML]{330001} 775153,69}            & {\color[HTML]{333333} 759684}       & {\color[HTML]{333333} 2,00}          & {\color[HTML]{333333} 0,04}          & {\color[HTML]{333333} 768544}          & {\color[HTML]{333333} 0,85}          & {\color[HTML]{333333} 0,47}          \\
			{\color[HTML]{330001} MDG-b\_7\_n500\_m50}    & {\color[HTML]{330001} 777232,87}            & {\color[HTML]{333333} 765951}       & {\color[HTML]{333333} 1,45}          & {\color[HTML]{333333} 0,06}          & {\color[HTML]{333333} 769574}          & {\color[HTML]{333333} 0,99}          & {\color[HTML]{333333} 0,50}          \\
			{\color[HTML]{330001} MDG-b\_8\_n500\_m50}    & {\color[HTML]{330001} 779168,75}            & {\color[HTML]{333333} 775018}       & {\color[HTML]{333333} 0,53}          & {\color[HTML]{333333} 0,04}          & {\color[HTML]{333333} 771452}          & {\color[HTML]{333333} 0,99}          & {\color[HTML]{333333} 0,54}          \\
			{\color[HTML]{330001} MDG-b\_9\_n500\_m50}    & {\color[HTML]{330001} 774802,19}            & {\color[HTML]{333333} 770633}       & {\color[HTML]{333333} 0,54}          & {\color[HTML]{333333} 0,05}          & {\color[HTML]{333333} 769227}          & {\color[HTML]{333333} 0,72}          & {\color[HTML]{333333} 0,50}          \\
			{\color[HTML]{330001} MDG-b\_10\_n500\_m50}   & {\color[HTML]{330001} 774961,31}            & {\color[HTML]{333333} 766173}       & {\color[HTML]{333333} 1,13}          & {\color[HTML]{333333} 0,05}          & {\color[HTML]{333333} 769320}          & {\color[HTML]{333333} 0,73}          & {\color[HTML]{333333} 0,45}          \\
			{\color[HTML]{330001} MDG-a\_31\_n2000\_m200} & {\color[HTML]{330001} 114139}               & {\color[HTML]{333333} 113825}       & {\color[HTML]{333333} 0,28}          & {\color[HTML]{333333} 0,41}          & {\color[HTML]{333333} 113008}          & {\color[HTML]{333333} 0,99}          & {\color[HTML]{333333} 4,21}          \\
			{\color[HTML]{330001} MDG-a\_32\_n2000\_m200} & {\color[HTML]{330001} 114092}               & {\color[HTML]{333333} 113268}       & {\color[HTML]{333333} 0,72}          & {\color[HTML]{333333} 0,46}          & {\color[HTML]{333333} 112858}          & {\color[HTML]{333333} 1,08}          & {\color[HTML]{333333} 4,23}          \\
			{\color[HTML]{330001} MDG-a\_33\_n2000\_m200} & {\color[HTML]{330001} 114124}               & {\color[HTML]{333333} 113223}       & {\color[HTML]{333333} 0,79}          & {\color[HTML]{333333} 0,43}          & {\color[HTML]{333333} 113062}          & {\color[HTML]{333333} 0,93}          & {\color[HTML]{333333} 4,70}          \\
			{\color[HTML]{330001} MDG-a\_34\_n2000\_m200} & {\color[HTML]{330001} 114203}               & {\color[HTML]{333333} 113306}       & {\color[HTML]{333333} 0,79}          & {\color[HTML]{333333} 0,38}          & {\color[HTML]{333333} 112887}          & {\color[HTML]{333333} 1,15}          & {\color[HTML]{333333} 4,55}          \\
			{\color[HTML]{330001} MDG-a\_35\_n2000\_m200} & {\color[HTML]{330001} 114180}               & {\color[HTML]{333333} 113838}       & {\color[HTML]{333333} 0,30}          & {\color[HTML]{333333} 0,37}          & {\color[HTML]{333333} 112966}          & {\color[HTML]{333333} 1,06}          & {\color[HTML]{333333} 4,87}          \\
			{\color[HTML]{330001} MDG-a\_36\_n2000\_m200} & {\color[HTML]{330001} 114252}               & {\color[HTML]{333333} 113220}       & {\color[HTML]{333333} 0,90}          & {\color[HTML]{333333} 0,39}          & {\color[HTML]{333333} 113002}          & {\color[HTML]{333333} 1,09}          & {\color[HTML]{333333} 4,81}          \\
			{\color[HTML]{330001} MDG-a\_37\_n2000\_m200} & {\color[HTML]{330001} 114213}               & {\color[HTML]{333333} 113447}       & {\color[HTML]{333333} 0,67}          & {\color[HTML]{333333} 0,38}          & {\color[HTML]{333333} 112946}          & {\color[HTML]{333333} 1,11}          & {\color[HTML]{333333} 4,39}          \\
			{\color[HTML]{330001} MDG-a\_38\_n2000\_m200} & {\color[HTML]{330001} 114378}               & {\color[HTML]{333333} 113427}       & {\color[HTML]{333333} 0,83}          & {\color[HTML]{333333} 0,37}          & {\color[HTML]{333333} 113016}          & {\color[HTML]{333333} 1,19}          & {\color[HTML]{333333} 4,62}          \\
			{\color[HTML]{330001} MDG-a\_39\_n2000\_m200} & {\color[HTML]{330001} 114201}               & {\color[HTML]{333333} 113589}       & {\color[HTML]{333333} 0,54}          & {\color[HTML]{333333} 0,38}          & {\color[HTML]{333333} 112761}          & {\color[HTML]{333333} 1,26}          & {\color[HTML]{333333} 4,34}          \\
			{\color[HTML]{330001} MDG-a\_40\_n2000\_m200} & {\color[HTML]{330001} 114191}               & {\color[HTML]{333333} 113398}       & {\color[HTML]{333333} 0,69}          & {\color[HTML]{333333} 0,39}          & {\color[HTML]{333333} 112959}          & {\color[HTML]{333333} 1,08}          & {\color[HTML]{333333} 4,61}          \\ \bottomrule
		\end{tabular}%
	}
	\caption{Resultados ILS}
	\label{AMallo}
\end{table}




\[
\begin{array}{r|*{4}{r}}{Algoritmo}&Desv&Tiempo\\\hline


{}Greedy&1,36&0,46\\

{}BL&1,07&0.49\\

{}ES&1,15&0,17\\

{}BMB&0,72&0,24\\

{}ILS&0,55&0,16\\

{}ILS\text{-}ES&0,70&1,78
\end{array}
\label{glob}
\]


\newpage
\section{Análisis de Resultados}

\begin{itemize}

\item Observando la tabla de resultados globales \ref{glob} anterior, el algoritmo \textit{Greedy} es con el que peor resultados se obtienen. El resto de algoritmos considerados tienen una desviación que se encuentra en un rango de valores muy óptimo y ajustado.

\item Por otro lado, todos los algoritmos son muy eficientes, con un tiempo de ejecución de media por debajo de 1 segundo, a excepción de \textit{ILS-ES} que casi dobla este valor.


\item Dejando el \textit{Greedy} de lado, obtenemos dos grupos de algoritmos, los que están basados en trayectorias simples (\textit{BL y ES}) y los que están basados en trayectorias múltiples (\textit{BMB, ILS e ILS-ES}), caracterizados por ejecutar las técnicas de trayectorias simples varias veces.

Teniendo en cuenta este punto, los resultados obtenidos cuadran con el razonamiento teórico, ya que \textit{BMB, ILS e ILS-ES} reducen en torno al 40\% la desviación obtenida de las técnicas basadas en trayectorias simples. 
\end{itemize}








Recordemos los casos seleccionados para la ejecución de los algoritmos:
Mostramos los resultados obtenidos en gráficas diferentes según el grupo al que pertenezca cada instancia para reflejar mejor las diferencias:



\begin{itemize}
	\item \texttt{GKD-c}: Conjunto de datos, referentes al grupo de distancias Euclideas con $n=500$, $m=50$,  en el que mejor resultados se han obtenido, tanto los algoritmos basados en técnicas de trayectorias simples como múltiples llegan al mejor coste.  Hay convergencia con todos algoritmos en todas las instancias, de hecho, se aprecia que todas las líneas están superpuestas, obteniendo prácticamente una desviación nula.
	
	
	\begin{figure}[H]
		\centering
		\includegraphics[scale=0.62]{img/gkdc.png}
	\end{figure}
	
	
	\newpage 
	
	\item \texttt{MDG-a}: Conjunto de datos correspondiente al grupo de distancias enteras y tamaño de $n=2000$ y $m=200$. En todas las instancias los costes obtenidos por los algoritmos basados en técnicas de trayectorias múltiples superan a los que están basados en técnicas de trayectorias simples. 
	Se aprecia como la línea de \textit{ILS} se encuentra por encima del resto, mientras la roja y amarilla, pertenecientes a los algoritmos \textit{BL y ES}, se encuentran por debajo.
	
	

	
	
	\begin{figure}[H]
		\centering
		\includegraphics[scale=0.62]{img/mdga.png}
	\end{figure}
	
	
	
	\item \texttt{MDG-b}: Conjunto de datos correspondiente al grupo de distancias reales y tamaño $n=500$ y $m=50$. 
	Al igual que en el grupo de instancias anterior, los algoritmos basados en trayectorias simples (\textit{BL} y \textit{ES}) se encuentran por debajo del resto.
	
	Si observamos el trazado por cada uno de los algoritmos, observamos que se presentan altibajos o fluctuaciones en las mismas instancias. Esto es debido a la presencia de óptimos locales, haciendo costoso que nuestros algoritmos puedan salir de estos. 
	
	

	\begin{figure}[H]
		\centering
		\includegraphics[scale=0.62]{img/mdgb.png}
	\end{figure}
	
\end{itemize}

\newpage

Un aspecto diferente en los algoritmos implementados radica en el número de soluciones usadas en el proceso de optimización. \textit{Enfriamiento Simulado} o \textit{Búsqueda Local} utilizan una sola solución en el proceso de búsqueda. \\



Por otro lado, técnicas como \textit{Búsqueda Multiarranque Básica} o \textit{Búsqueda Local Reiterada} optimizan un conjunto de soluciones iniciales creadas de forma aleatoria por lo que se obtienen soluciones de más calidad debido a la ventaja que supone explorar diferentes áreas del espacio de búsqueda.

Debido a la posible presencia de óptimos locales, algoritmos como \textit{Búsqueda Local Reiterada} (\textit{ILS}) presentan un factor de mutación brusco, mejorando en gran medida los resultados obtenidos.







\subsection{Desviación}
Por último, reflejados en el siguiente gráfico en red, mostramos la desviación obtenida de todas las instancia tras la ejecución de cada algoritmo. Como bien sabemos, cuanto menor sea la desviación, mejor será el coste obtenido con nuestro algoritmo. Por tanto, el algoritmo que presente un mayor área bajo su trazado, será peor.




\begin{figure}[H]
	\centering
	\includegraphics[scale=0.6]{img/desv.png}
	
\end{figure}



En resumen, podemos considerar que tanto la eficacia como la eficiencia de todos nuestros algoritmos es muy buena por tener una desviación y un tiempo muy pequeños.

Con los que mejor resultados se obtienen son con los algoritmos basados en técnicas de trayectorias múltiples, especialmente el algoritmo de \textit{Búsqueda Local Reiterada} (\textit{ILS}) ya que en media obtiene soluciones más cercanas al mejor valor conocido.









\newpage
\begin{thebibliography}{X} 
	
-\href{https://sci2s.ugr.es/sites/default/files/files/Teaching/GraduatesCourses/Metaheuristicas/Sem01-Problemas-MHs-2019-20.pdf}{Seminario 1}\\
	
	-\href{https://sci2s.ugr.es/sites/default/files/files/Teaching/GraduatesCourses/Metaheuristicas/Sem02-Problemas-BusquedaLocal-MHs-19-20.pdf}{Seminario 2}\\
	
		-\href{https://sci2s.ugr.es/sites/default/files/Sem03-Problemas-Poblaciones-MHs-19-20.pdf}{Seminario 3}\\
		
		-\href{https://sci2s.ugr.es/sites/default/files/files/Teaching/GraduatesCourses/Metaheuristicas/Sem04-Problemas-Trayectorias-MHs-19-20.pdf}{Seminario 4}\\
	
	-\href{https://sci2s.ugr.es/sites/default/files/files/Teaching/GraduatesCourses/Metaheuristicas/Guion\%20P3a\%20Trayectorias\%20MDP\%20MHs\%202019-20.pdf}{Guión de prácticas}




\end{thebibliography}


\end{document}

